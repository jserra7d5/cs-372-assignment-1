\documentclass[12pt]{article}
\usepackage[utf8]{inputenc}
\usepackage[margin=1in]{geometry}
\usepackage{graphicx}
\usepackage{float}
\usepackage{listings}
\usepackage{xcolor}
\usepackage{url}
\usepackage{fancyhdr}


\lstset{
    basicstyle=\ttfamily\footnotesize,
    backgroundcolor=\color{gray!10},
    frame=single,
    breaklines=true,
    numbers=left,
    numberstyle=\tiny,
    showstringspaces=false
}


\pagestyle{fancy}
\fancyhf{}
\rhead{Socket Programming Assignment}
\lhead{CS 372}
\cfoot{\thepage}

\title{Socket Programming Assignment}
\author{Joseph Serra}
\date{\today}

\begin{document}

\maketitle

\section{Overview}

Code repository: \url{https://github.com/jserra7d5/cs-372-assignment-1}\\

This assignment demonstrates socket programming in Python using the raw socket API. Three programs were implemented to showcase different aspects of network communication:

\begin{enumerate}
    \item \textbf{simple\_get.py} - Basic HTTP GET client for small files
    \item \textbf{large\_file\_get.py} - HTTP GET client with loop for large files
    \item \textbf{simple\_server.py} - Simple HTTP server
\end{enumerate}

All programs use the Python socket API as required.

\section{Implementation Details}

\subsection{Simple GET Client (simple\_get.py)}

This program connects to \texttt{gaia.cs.umass.edu} and retrieves a small HTML file using a single \texttt{recv()} call.

\textbf{Usage:}
\begin{lstlisting}[language=bash]
python3 simple_get.py
\end{lstlisting}

\textbf{Screenshot of simple\_get.py execution:}

\begin{figure}[H]
    \centering
    \includegraphics[width=0.8\textwidth]{simple_get_screenshot.png}
    \caption{Execution of simple\_get.py showing HTTP GET request and response}
    \label{fig:simple_get}
\end{figure}

\subsection{Large File GET Client (large\_file\_get.py)}

This program demonstrates handling larger files by implementing a receive loop that continues until the server closes the connection.

\textbf{Usage:}
\begin{lstlisting}[language=bash]
python3 large_file_get.py
\end{lstlisting}

\textbf{Screenshot of large\_file\_get.py execution:}

\begin{figure}[H]
    \centering
    \includegraphics[width=0.8\textwidth]{large_get_screenshot_1.png}
    \caption{(Pt. 1) Execution of large\_file\_get.py showing chunked file retrieval}
    \label{fig:large_get_1}
\end{figure}

\begin{figure}[H]
    \centering
    \includegraphics[width=0.8\textwidth]{large_get_screenshot_2.png}
    \caption{(Pt. 2) Execution of large\_file\_get.py showing chunked file retrieval}
    \label{fig:large_get_2}
\end{figure}

\subsection{Simple HTTP Server (simple\_server.py)}

This program creates a basic HTTP server that listens on localhost and serves a simple HTML response to web browsers.

\textbf{Usage:}
\begin{lstlisting}[language=bash]
python3 simple_server.py
\end{lstlisting}

Then open browser to: \texttt{http://127.0.0.1:8080}

\textbf{Screenshot of simple\_server.py execution:}

\begin{figure}[H]
    \centering
    \includegraphics[width=0.8\textwidth]{server_output.png}
    \caption{Execution of simple\_server.py showing server startup and client connections}
    \label{fig:server_terminal}
\end{figure}

\textbf{Screenshot of browser accessing the server:}

\begin{figure}[H]
    \centering
    \includegraphics[width=0.8\textwidth]{browser_output.png}
    \caption{Browser view of the HTTP server response}
    \label{fig:server_browser}
\end{figure}


\end{document}